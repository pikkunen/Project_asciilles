\documentclass[12pt, a4paper]{article}
%\usepackage[swedish]{babel}
%\usepackage[T1]{fontenc}
%\usepackage[utf8]{inputenc}
\usepackage{graphicx}
\usepackage{url}
\usepackage{amsmath}
\usepackage{fancyheadings}
\newcommand{\HRule}{\rule{\linewidth}{0.6mm}}

\addtolength{\textheight}{20mm}
\addtolength{\voffset}{-5mm}
\renewcommand{\sectionmark}[1]{\markleft{#1}}


% appendices, \appitem och \appsubitem är för bilagor
\newcounter{appendixpage}

\newenvironment{appendices}{
	\setcounter{appendixpage}{\arabic{page}}
	\stepcounter{appendixpage}
}{
}

\newcommand{\appitem}[2]{
	\stepcounter{section}
	\addtocontents{toc}{\protect\contentsline{section}{\numberline{\Alph{section}}#1}{\arabic{appendixpage}}}
	\addtocounter{appendixpage}{#2}
}

\newcommand{\appsubitem}[2]{
	\stepcounter{subsection}
	\addtocontents{toc}{\protect\contentsline{subsection}{\numberline{\Alph{section}.\arabic{subsection}}#1}{\arabic{appendixpage}}}
	\addtocounter{appendixpage}{#2}
}



%%%%%%%%%%%%%%%%%%%% Change These %%%%%%%%%%%%%%%%%%%%%%%%%%%%
\newcommand{\names}{Elli \textsc{Virtanen}\\
					David \textsc{Pikas}\\
					Wietze \textsc{Schelhaas}}
\newcommand{\teacher}{Carlos \textsc{Penichet}\\
					  Dave \textsc{Clarke}\\
					  Tjark \textsc{Weber}}
\newcommand{\rapportnamn}{Projekt ASCIIlles}
%\newcommand{\undertitel}{Och en eventuell cool undertitel}
\newcommand{\kursnamn}{Program Design and Data Structures 20c}
%%%%%%%%%%%%%%%%%%%%%%%%%%%%%%%%%%%%%%%%%%%%%%%%%%%%%%%%%%%%%%

\begin{document}

\begin{titlepage}
	\begin{center}

	\includegraphics{uu_logo.png} \\[1.5cm]
	\Large{\kursnamn}\\ 
	\HRule \\[0.5cm]
	\textsc{\huge{\rapportnamn}}\\
	%\Large{\undertitel} \\
	\HRule \\[0.5cm]
	\vspace{15 mm}


\begin{minipage}{0.4\textwidth}
	\begin{flushleft} \large
		\emph{Authors:}\\
		\names\\
	\end{flushleft}
\end{minipage}
\hfill
\begin{minipage}{0.4\textwidth}
	\begin{flushright} \large
		\emph{Supervisors:}\\
		\teacher\\
	\end{flushright}
\end{minipage}

\vfill
\large Bachelor Programme in Computer Science

\textsc{\large Uppsala Universitet \\ \today}

\end{center}
\end{titlepage}

%%%% stycket är snott från Umeå Universitets rapportmall
% fixar sidfot
\lfoot{\footnotesize{\names}}
\rfoot{\footnotesize{\today}}
\lhead{\sc\footnotesize\rapportnamn}
\rhead{\nouppercase{\sc\footnotesize\leftmark}}
\pagestyle{fancy}
\renewcommand{\headrulewidth}{0.2pt}
\renewcommand{\footrulewidth}{0.2pt}
% skapar innehållsförteckning.
% Tänk på att köra latex 2ggr för att uppdatera allt
\pagenumbering{roman}


\tableofcontents
	
% och lägger in en sidbrytning
\newpage

\pagenumbering{arabic}

\section{Problemspecifikation}
		% \emph innebär emphasize, d.v.s. betona eller framhåll -> kursiv stil
		\emph{I detta avsnitt beskrivs laborationen, dels som en sammanfattning
			över dess syfte, men även en koppling till orginalspecifikationen
			tas upp.}

		% lägg in en underrubrik (\subsection -> spillutrymme)
		\subsection{Problemsammanfattning}

		\subsection{Orginalspecifikation}
			Specifikationen i sin helhet finns ...

			% nytt stycke skapas när du har en, eller flera, tomma rader
			någonstans i föregående stycke

	\section{Åtkomst och användarhandledning}
		\emph{I detta avsnitt beskrivs de filer, med tillhörande sökvägar,
			som ingår i lösningen, samt korta beskrivningar om syftet med
			varje fil. Dessutom beskrivs hur man ska handskas med den
			implementerade lösningen till problemet, i form av instruktioner
			för hur körning och kompilering av lösningen sker.}
	
		\subsection{Filer som ingår i lösningen}
			
			
		\subsection{Kompilering och körning}
			

	\section{Systembeskrivning}
		\emph{I detta avsnitt beskrivs systemet i sin helhet mer i detalj.
			Datastrukturer och annan intern representation som är central
			för uppgiftens lösning behandlas. Dessutom beskrivs de olika
			komponenternas relationer till varandra.}

		%Dessa rubriker behöver inte finnas med, men något i denna stil
		%bör det kanske vara
		\subsection{Systemöversikt}

		\subsection{Exekveringsflöde}
			
		\subsection{Övriga systemdetaljer}

	\section{Algoritmbeskrivning}
		\emph{I detta avsnitt beskrivs de algoritmer som anses som
			icke-triviala i uppgiften.}
		
		\subsection{Algoritm 1}
			
		\subsection{Algoritm 2}	
	
	\section{Lösningens begränsningar}
		\emph{I detta avsnitt beskrivs alla begränsningar som lösningen av
			uppgiften innehåller. Detta innefattar även funna begränsningar
			som strider mot specifikationen.}
		
		SKRIV TEXT HÄR...

	\section{Problem och reflektioner}
		\emph{Här presenteras egna tankar kring uppgiften som sådan samt de
			problem som uppstått under arbetets gång.}
		
		OCH HÄR..

	\section{Testkörningar}
		\emph{I detta avsnitt visas ett antal tester som utförts med programmet.
			Samtliga tester är kommenterade.}
		
		GLÖM INTE ATT KOMMENTERA TESTER!
		
		\begin{footnotesize}
			\begin{verbatim}
				Utdata kan lämpligtvis återges så här, men tänk på att
				        kommentera dina tester. Lägg märke till att rad-
				brytningarna och mellanslagen bevaras!
			\end{verbatim}
		\end{footnotesize}
		

	% här börjar alla bilagor. Denna måste finnas med även om bara
	% bilagor anges i \begin{appendices} ... \end{appendices}
	\appendix

	\section{Bilaga 1}
	\ldots{}ligger direkt i dokumentet

	% bilagor, t.ex. källkod. En tom extrasida kommer att skrivas ut för
	% att få alla sidnummer att stämma
	\begin{appendices}
		\appitem{Källkod}{0}
		\appsubitem{\texttt{minfil.c}}{2}
		\appsubitem{\texttt{minfil.h}}{1}
		\appitem{En bilaga på 3 sidor}{3}
	\end{appendices}

\begin{thebibliography}{references}
\bibitem{Overleaf}\url{https://www.overleaf.com}
\end{thebibliography}

\end{document}