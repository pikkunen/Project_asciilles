\documentclass[12pt, a4paper]{article}
%\usepackage[swedish]{babel}
%\usepackage[T1]{fontenc}
%\usepackage[utf8]{inputenc}
\usepackage{graphicx}
\usepackage{url}
\usepackage{amsmath}
\usepackage{fancyheadings}
\newcommand{\HRule}{\rule{\linewidth}{0.6mm}}

\addtolength{\textheight}{20mm}
\addtolength{\voffset}{-5mm}
\renewcommand{\sectionmark}[1]{\markleft{#1}}


% appendices, \appitem och \appsubitem är för bilagor
\newcounter{appendixpage}

\newenvironment{appendices}{
	\setcounter{appendixpage}{\arabic{page}}
	\stepcounter{appendixpage}
}{
}

\newcommand{\appitem}[2]{
	\stepcounter{section}
	\addtocontents{toc}{\protect\contentsline{section}{\numberline{\Alph{section}}#1}{\arabic{appendixpage}}}
	\addtocounter{appendixpage}{#2}
}

\newcommand{\appsubitem}[2]{
	\stepcounter{subsection}
	\addtocontents{toc}{\protect\contentsline{subsection}{\numberline{\Alph{section}.\arabic{subsection}}#1}{\arabic{appendixpage}}}
	\addtocounter{appendixpage}{#2}
}



%%%%%%%%%%%%%%%%%%%% Change These %%%%%%%%%%%%%%%%%%%%%%%%%%%%
\newcommand{\names}{Elli \textsc{Virtanen}\\
					David \textsc{Pikas}\\
					Wietze \textsc{Schelhaas}}
\newcommand{\teacher}{Carlos \textsc{Penichet}\\
					  Dave \textsc{Clarke}\\
					  Tjark \textsc{Weber}}
\newcommand{\rapportnamn}{Project ASCIIlles}
%\newcommand{\undertitel}{Och en eventuell cool undertitel}
\newcommand{\kursnamn}{Program Design and Data Structures 20c}
%%%%%%%%%%%%%%%%%%%%%%%%%%%%%%%%%%%%%%%%%%%%%%%%%%%%%%%%%%%%%%

\begin{document}

\begin{titlepage}
	\begin{center}
	\includegraphics[scale=0.35]{uu_ascii.png} \\[1.5cm]
	\Large{\kursnamn}\\ 
	\HRule \\[0.5cm]
	\textsc{\huge{\rapportnamn}}\\
	%\Large{\undertitel} \\
	\HRule \\[0.5cm]
	\vspace{15 mm}


\begin{minipage}{0.4\textwidth}
	\begin{flushleft} \large
		\emph{Authors:}\\
		\names\\
	\end{flushleft}
\end{minipage}
\hfill
\begin{minipage}{0.4\textwidth}
	\begin{flushright} \large
		\emph{Supervisors:}\\
		\teacher\\
	\end{flushright}
\end{minipage}

\vfill
\large Bachelor Programme in Computer Science

\textsc{\large Uppsala Universitet \\ \today}

\end{center}
\end{titlepage}

%%%% stycket är snott från Umeå Universitets rapportmall
% fixar sidfot
\lfoot{\footnotesize{\names}}
\rfoot{\footnotesize{\today}}
\lhead{\sc\footnotesize\rapportnamn}
\rhead{\nouppercase{\sc\footnotesize\leftmark}}
\pagestyle{fancy}
\renewcommand{\headrulewidth}{0.2pt}
\renewcommand{\footrulewidth}{0.2pt}
% skapar innehållsförteckning.
% Tänk på att köra latex 2ggr för att uppdatera allt
\pagenumbering{roman}


\tableofcontents

% och lägger in en sidbrytning
\newpage

\pagenumbering{arabic}

	\section{Project description}

		Project ASCIIlles is a short project in the course Program Design and Data Structures teached at Uppsala University during the fall semester 2015 and the spring semester 2016. The subject of this project is to create a program called ASCIIfy which converts an image to ASCII-art. The side effects are that the team gains experience in working with projects and using modules made by others.

		%\emph{I detta avsnitt beskrivs laborationen, dels som en sammanfattning över dess syfte, men även en koppling till orginalspecifikationen			tas upp.}

		% lägg in en underrubrik (\subsection -> spillutrymme)
		\subsection{Program description }

		ACSIIfy scans an image and splits it into smaller segments. The average greyscale value of every segment is calculated and represented as an ASCII-character in the exported into the terminal or as a HTML-file. This program relies on the use of JuicyPixels.

		%\subsection{Orginalspecifikation}
			%Specifikationen i sin helhet finns ... nytt stycke skapas när du har en, eller flera, tomma rader någonstans i föregående stycke

	\section{User manual}

	The main.hs file is, as the name suggests the file that will actually run the program. The program takes one to three command-line arguments. The first is the file path to the image that is to be turned into ascii. This argument is mandatory. The other two are the output file and the scale. If no output file is provided, the program will print the result in the terminal. If the output file has the .html extension, some html and css will be added to make it look nicer (namely, line breaks, smaller font-size, monospaced font and better horizontal letter spacing). Scale should be given as a natural number, the bigger the number the smaller the returned ASCII will be. If no scale is given a standard scale will be used.

	To run this program you need the Haskell module JuicyPixels, which can be installed with cabal or downloaded from \url{http://hackage.haskell.org/package/JuicyPixels}.


		%\emph{I detta avsnitt beskrivs de filer, med tillhörande sökvägar,som ingår i lösningen, samt korta beskrivningar om syftet med	varje fil. Dessutom beskrivs hur man ska handskas med den implementerade lösningen till problemet, i form av instruktioner för hur körning och kompilering av lösningen sker.}

		\subsection{Files included in the program }

		main.hs | Contains the main function and the functions responsible for loading and writing files.

		imageTo2dlist.hs | Contains the functions to convert an image into a list that represents the pixels of the image.

		asciilate.hs | Contains the functions to convert a list of pixels into ascii-art.

		\subsection{How to run}

		To run, run the following line in the terminal
		\begin{footnotesize}
			\begin{verbatim}
				runhaskell main.hs filepath [destination] [scale]
			\end{verbatim}
		\end{footnotesize}

		Example: To run the file dave.jpeg and export it as a HTML-file with the greatest possible resolution one should run the following in the terminal:

		\begin{footnotesize}
			\begin{verbatim}
				runhaskell main.hs dave.jpeg dave.html 1
			\end{verbatim}
		\end{footnotesize}

	\section{Program documentation}

	When the program starts, the firsts thing it does is to read the image or gif from the specified filepath. It does this by calling the functions readImage or readGifImages, which are funtions provided by the Haskell library JuicyPixels.

	Then the program converts colored image/images to grayscale. Giving every pixel has a value between 0 and 255, where 0 is black and 255 is white and the numbers between are different shades of grey.

	The program determines what symbol is to replace the pixels with by calculating the average greyscale value of a chunk of pixels.The chunk size is determined by the scale argument, for instance if you specified a scale of 2 the program would take a chunk containing 2x2 pixels. Chunks with lower values are darker so it should be replaced with a symbol containing alot of black.

	This function is recursively applied to rows of pixels from the image and returns a string for evey row.  These rows are combined together and written into the specified file, if the user hasen't specified a location to write the file then the program will print the ASCII-art straight into the terminal.

		%\emph{I detta avsnitt beskrivs systemet i sin helhet mer i detalj. Datastrukturer och annan intern representation som är central för uppgiftens lösning behandlas. Dessutom beskrivs de olika komponenternas relationer till varandra.}

		%Dessa rubriker behöver inte finnas med, men något i denna stil
		%bör det kanske vara
		%\subsection{Systemöversikt}

		%\subsection{Exekveringsflöde}

		%\subsection{Övriga systemdetaljer}


	\section{Algoritm description}

	The two most important algorithms that ASCIIfy uses are ImageTo2dList and Asciilate. These algorithms will be described below.
		%\emph{I detta avsnitt beskrivs de algoritmer som anses som icke-triviala i uppgiften.}

		\subsection{Algoritm 1}

		\subsection{Algoritm 2}

	\section{The solutions limitations}

	ASCIIfy is designed to support all types of images that the Haskell library JuicyPixels (used for loading and storing images) supports, which are PNG-files, JPEG-files and GIF-files. The program is not equipped to handle files in the CMYK colorspace.

	The support of GIF-files is somewhat limited, the framerate for animated GIFs is constant och can thus appear awkward.

	ASCIIfy is also a bit slow, so high resolution images are slow to run. This could be optimized in ASCIIfy 2.0 by changing the algorithm for deciding which character should substitute a chunk of pixels. In this version the program calculates the mean greyscale value for the chunk of pixels, but in the future a new algorithm could use fewer values. 

		%\emph{I detta avsnitt beskrivs alla begränsningar som lösningen av
		%	uppgiften innehåller. Detta innefattar även funna begränsningar
		%	som strider mot specifikationen.}

		%SKRIV TEXT HÄR...

	\section{Problems and reflections}

		The greatest challenge durning this project was not the programming itself, but to understand and to use JuicyPixels. A lot of time was spent on researching, installing, reinstalling and rereinstalling libraries for handling images in Haskell. but i t was also a good experience for the group to learn how to do that kind of research. 

		%\emph{Här presenteras egna tankar kring uppgiften som sådan samt de problem som uppstått under arbetets gång.}


	\section{Test cases}
		%\emph{I detta avsnitt visas ett antal tester som utförts med programmet.Samtliga tester är kommenterade.}

		%GLÖM INTE ATT KOMMENTERA TESTER!

		Tests:

		ImageTo2DList:
		\begin{itemize}
			\item{fillTransparency y a 255 should result in a value
			      between 0 and 255 if y and a are also between 0
						and 255}
			\item{Given a 2DList, calling list2DtoImage and then
			      ImageTo2DList should result in the original list
						(if the original list isn't empty)}
			\item{fillTransparency 0 127 255 should equal 128}
			\item{fillTransparency 0 0 255 should equal 255}
		\end{itemize}

		Asciilate:
		\begin{itemize}
			\item{Turning a 2DList of symbols into a 2DList of
			      grey values and then calling asciilate on that
						should result in the original 2DList}
			\item{The scale paramater should result in the result
			      decreasing in size relative to its input in the
						way described in its specification}
			\end{itemize}

		%\begin{footnotesize}
			%\begin{verbatim}
				%Utdata kan lämpligtvis återges så här, men tänk på att kommentera dina tester. Lägg märke till att rad-
				%brytningarna och mellanslagen bevaras!
			%\end{verbatim}
		%\end{footnotesize}


	% här börjar alla bilagor. Denna måste finnas med även om bara
	% bilagor anges i \begin{appendices} ... \end{appendices}
	\appendix

	\section{Appendixes}
	%\ldots{}ligger direkt i dokumentet

	% bilagor, t.ex. källkod. En tom extrasida kommer att skrivas ut för
	% att få alla sidnummer att stämma
	\begin{appendices}
		\appitem{Some code}{0}
		\appsubitem{\texttt{minfil.c}}{2}
		\appsubitem{\texttt{minfil.h}}{1}
		\appitem{En bilaga på 3 sidor}{3}
	\end{appendices}

\begin{thebibliography}{references}
\bibitem{JuicyPixels}\url{http://hackage.haskell.org/package/JuicyPixels}
\end{thebibliography}

\end{document}
